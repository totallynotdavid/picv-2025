\documentclass[a4paper,11pt]{article}
\usepackage{latexsym}
\usepackage[utf8]{inputenc}
\usepackage[spanish]{babel}
\usepackage{graphicx}
\usepackage{booktabs}

\title{@title}
\author{@author}
\frenchspacing

\begin{document}
\maketitle

\section*{Introducción}
Este reporte preliminar de tsunami de origen lejano ha sido elaborado de forma automática.

\section*{Parámetros Hipocentrales y Mecanismo Focal}
\begin{table}[h!]
\centering
\begin{tabular}{lcc}
\toprule
Parámetro & & Valor \\
\midrule
Latitud     & & @lat\ $^\circ$ \\
Longitud    & & @lon\ $^\circ$ \\
Profundidad & & @depth\ km \\
Magnitud    & & @magnitude \\
\midrule
Strike      & & @strike\ $^\circ$ \\
Dip         & & @dip\ $^\circ$ \\
Rake        & & @rake\ $^\circ$ \\
\bottomrule
\end{tabular}
\caption{Parámetros hipocentrales y mecanismo focal del terremoto.}
\end{table}

\section*{Tsunami en Estaciones Seleccionadas}
\begin{table}[h!]
\centering
\begin{tabular}{lcc}
\toprule
Estación & Tiempo de Arribo & Máximo (m) \\
\midrule
@station1_name & @station1_time & @station1_max \\
@station2_name & @station2_time & @station2_max \\
@station3_name & @station3_time & @station3_max \\
\bottomrule
\end{tabular}
\caption{Tiempos de arribo y amplitudes máximas en estaciones seleccionadas.}
\end{table}

\end{document}
